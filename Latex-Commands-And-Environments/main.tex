\documentclass{article}

\usepackage{enumerate} % Used for customizing ordered lists

\title{Latex Commands and Environments}
\author{Irfan}
\date{28 September 2023}

% Commands that have a \begin and \end are called "Environments"
\begin{document}

% Commands that don't neeed \begin and \end are simply known as "commands"
\maketitle

% "abstract" is another example of an environment

\begin{abstract}
  This is the document abstract. It contains a brief summary of the document.
\end{abstract}

% "footnote" is a command
This is the document body\footnote{Body is the main content}.

We can set custome number to footnotes\footnote[5]{A footnote}.

\section{Unordered Lists}
% We can make unordered lists using the "itemize" environment
\begin{itemize}
  \item Apple
  \item Orange
  \item Grape
  \begin{itemize} % This is a nested list
    \item Pink Grape
    \item Purple Grape
  \end{itemize}
\end{itemize}

\section{Customized Unordered Lists}
% We can put custom markers in unordered lists
\begin{itemize}
  \item[*] Apple
  \item[*] Orange
  \item[*] Grape
\end{itemize}

\section{Ordered Lists}
% The "enumerate" environment is used for creating ordered lists.
% There's also an "enumerate" package for customizing ordered lists.
\begin{enumerate}
  \item Red
  \item Green
  \item Blue
  \begin{enumerate} % Nested list
    \item Bird
    \item Berry
  \end{enumerate}
\end{enumerate}

\section{Customized ordered lists}
First List:
\begin{enumerate}[i.] % 'i' optional argument will render roman numerals
  \item Red
  \item Green
  \item Blue
\end{enumerate}

Second List:
\begin{enumerate}[i)] % This will use parentheses instead of dots
  \item Red
  \item Green
  \item Blue
\end{enumerate}

Third List:
\begin{enumerate}[A)] % 'A' optional argument will render capital characters
  \item Red
  \item Green
  \item Blue
\end{enumerate}

\section{Tables with default columns}
% Latex tables can be created using the "tabular" environment.
% Tables have three default columns. Right (r), Center (c) and Left (l).
% We can customize column vertical borders using the pipe '|' operator.
% NOTE: Tables created using "tabular" are INLINE
\begin{tabular}{ |r| c l } % Only right column will have vertical border
  \hline % This will add horizontal border
  Column heading 1 & Column heading 2 & Column heading3 \\ % This line break is necessary
  \hline
  \hline % We can add multiple borders
  1 & 2 & 3
\end{tabular}
Some text.

\section{Separating Tables from text}
% To solved the above problem, we have to wrap our table inside
% the "table" environment.
Some text.
% 'h' optional argument means "here". It indicates the next available space
% Without the 'h' argument, Latex will put the table in any random free space, typically
% at the beginning or at the end of the page.
% Other arguments include top (t) and bottom (b). Sometimes, passing the argument isn't enough
% and we may have to force the placement using the '!' symbol with the argument like "t!".
\begin{table}[h]
  \centering % Center the table
  \begin{tabular}{ |r| c l }
    \hline
    Column heading 1 & Column heading 2 & Column heading3 \\ % This line break is necessary
    \hline
    \hline
    1 & 2 & 3
  \end{tabular}
  \caption{This is a table.}
\end{table}
Some more text.

\end{document}
