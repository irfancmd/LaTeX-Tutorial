\documentclass{article}

\usepackage{amsmath, amssymb, amsthm}

\title{Mathematics in Latex - Part 2}
\author{Irfan}
\date{3 November 2023}

% Required for using the "theorem" environment
% Note that we're adding section optional argument for making section number the primary number.
\newtheorem{theorem}{Theorem}[section]

% We're making the definition environment's numbers based on the thoeorem's number.
% So, A definition's numbers will continue from theorem's number.
\theoremstyle{definition}
\newtheorem{definition}[theorem]{Definition}

\theoremstyle{remark}
\newtheorem*{remark}{Remark} % Use * for omitting numbers

\begin{document}

\maketitle

\section{Sum symbol}
  \subsection{Inline math}
    $\sum_{n=0}^{\infty}\frac{1}{n^2+1}$ Some random text.

  % Inline math shouldn't exceed one-third of the \textwidth. If it exceeds
  % that, it's better to use display math.
  \subsection{Display math}
    \[\sum_{n=0}^{\infty}\frac{1}{n^2+1}\]

\section{Integral symbol}
  \subsection{Inline math}
    $\int_{0}^{1}x^2dx$ Some random text.

  \subsection{Display math}
    \[\int_{0}^{1}x^2dx\]

\section{Double integral symbol}
  % For triple integral, use the "iiint" command.
  \subsection{Inline math}
    $\iint_{s}x^2+ydxdy$ Some random text.

  \subsection{Display math}
    \[\iint_{s}x^2+ydxdy\]

\section{Limit symbol}
  \subsection{Inline math}
    $\lim_{x\to 1}x$ Some random text.

  \subsection{Display math}
    \[\lim_{x\to 1}x\]

\section{Misc symbols}
  \subsection{Inline math}
    $\max_{n\in  \{1,2,3\}}n, \min, \sup, \inf, \limsup, \liminf$ Some random text.

  \subsection{Display math}
    \[\max_{n\in  \{1,2,3\}}n\]

\section{Multiline math}
  % Use "multiline*" environment for omitting equation numbers
  \begin{multline}
    f(x) = 2x + 2x^2 + 3x^3 + 4xy \\
      + 2x + 2x^2 + 3x^3 + 4xy \\
      + 2x + 2x^2
  \end{multline}

\section{The align environment}
  % Use "&" where we want to align the next line
  % Note that spaces need to be aligned as well
  % Put "notag" command in the current line if we don't want that line to be numbered
  \begin{align}
    2x + 3y + 3z + w &= 2  & 3x + 4y + 1z &=4 \\
    x + 4y + 3z + w &= 3 & -x - y - z - w &= 3 \notag
  \end{align}

  \subsection{Align continued}
  \begin{align}
    f(x) &= 3x + 3y - 2x \\
    &= 3y
  \end{align}

\section{Some formatting commands}
  \subsection{Adding extra spaces}
  % Use "quad" command for adding extra space and "qquad" command for adding even more space.
  \[\frac{x-1}{x-1}=1 \quad \forall x\neq1\]

  % Use "," command for adding a tiny space.
  \[f(x)=\int\sin(x)\,dx\]

\section{Modifying parentheses}
  ( \big( \Big( \bigg( \Bigg(

\section{Math fonts}
  \subsection{Bold font}
  $\mathbf{a}$

  \subsection{Caligraphy font}
  % Works only for capital letters
  $\mathcal{T}$

  \subsection{Blackboard-bold font}
  % Works only for capital letters
  $\mathbb{R}$

  \subsection{Bar symbol}
  % Best for one character
  $\bar{a}$

  \subsection{Overline symbol}
  % Best for multiple characters
  $\bar{a}$
  $\overline{ab}$

  \subsection{Dot symbol}
  $\dot{f}$

  \subsection{Prime symbol}
  $f', f'', f'''$

\section{Matrices}
  \subsection{Default - no boundaries}
  \[
    A =
    \begin{matrix}
      1 & 2 & 3 & 4  \\
      5 & 6 & 7 & 8
    \end{matrix}
  \]

  \subsection{With parentheses}
  \[
    A =
    \begin{pmatrix}
      1 & 2 & 3 & 4  \\
      5 & 6 & 7 & 8
    \end{pmatrix}
  \]

  \subsection{With brackets}
  \[
    A =
    \begin{bmatrix}
      1 & 2 & 3 & 4  \\
      5 & 6 & 7 & 8
    \end{bmatrix}
  \]

  \subsection{With vertical bars}
  % Used for determinant
  \[
    A =
    \begin{vmatrix}
      1 & 2 & 3 & 4  \\
      5 & 6 & 7 & 8
    \end{vmatrix}
  \]

\section{Cases}
    % Use the "text" command for inserting text inside a math environment.
    \[
      |x| =
      \begin{cases}
        x & \text{ for } x > 0 \\
        0 & \text{ for } x = 0 \\
        -x & \text{ for } x = 0
      \end{cases}
    \]

\section{Theorems}
\begin{theorem}[My Theorem]
 A theorem.
\end{theorem}

\begin{definition}
 A definition.
\end{definition}

\begin{definition}
 Another definition.
\end{definition}

\begin{remark}
 A remark.
\end{remark}

\section{Proofs}
\begin{proof}
  A Proof
\end{proof}

\end{document}
