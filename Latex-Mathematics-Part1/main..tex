\documentclass{article}

% For using mathematical symbols in Latex, we need these
% three packags by Ameriacan Mathematical Society (AMS).
% "amssymb" is for symbols and "amsthm" is for theorems.
\usepackage{amsmath, amssymb, amsthm}

\title{Mathematics in Latex - Part 1}
\author{Irfan}
\date{29 September 2023}

\begin{document}

\maketitle

\section{Inline math}
This is an inline math: $f(x) = 2x + 5$
\newline This is how to write multiplication using dot(.): $g(x) = 2 \cdot x$
\newline This is how to write not equal: $5 + 3 \ne 10$
\newline This is how to write square root: $\sqrt{2} = 1.414213562\ldots$
\newline This is how to write cubic root: $\sqrt[3]{2} = 1.25992104\ldots$
\newline This is how to write a fraction: $\frac{5}{3}$

\section{Using Greek Alphabet}
This is how to write small Pi $\pi$.
\newline This is how to write capital Pi $\Pi$.

\section{Display math}
% We can put mathematical expression inside '\[ \]' to render them in
% a separate line. This is called a display math command.

Some text. \[f(x) = 2x + C\] Some more text.

\section{Sub and super scripts}
This is how to write sub scripts: $2_{2}$
\newline This is how to write super scripts: $2^{2}$

\section{Trigonometric functions}
This is how to write sine: $\sin(30)$
\newline This is how to write cosine: $\cos(30)$

\end{document}
