\documentclass{article}

% To isntall latex packages in Fedora, use the command: %
% sudo dnf install tex-<latex package name> %

% The area before begin{document} is called "preamble". The area after is called
% the main body.

\usepackage{csquotes} % Used for adding quotes
\usepackage{blindtext} % Used for adding dummy text. This is useful for testing.
\usepackage{comment} % Used for multi-line comments

\title{Skeleton of a Latex Document}
\author{Irfan}
\date{\today}

\begin{document}

\maketitle

\tableofcontents

This is the document body.

\begin{comment}
  Everything inside this block
  will be treated as comments and won't be rendered.
\end{comment}

\section{This is a section}
Some random text of this section.

\subsection{This is a subsection}
Some text

% If we don't want the section to be numbered, we can use \section*{}
\section*{This is a section without number}
More random text.

\subsection{This is a subsection}
Some text
\subsection*{This is a subsection without number}
Some text

% Note that this numbered section will have serial '2' because non-numbered
% sections aren't counted.
\section{This is the third Section}
Even more random text.

\section{Using the "csquotes" package}

He said, \textquote{I love Latex}

\section{Using the "blindtext" package}

\blindtext

\section{Adding paragraphs}
% Put a blank line to indicate a new paragraph.

\blindtext

This is another paragraph.

\section{Adding paragraphs (Alternative approach)}

This is a paragraph. \par This is anothe paragraph.

\section{Adding paragraph without indentation}
This is a paragraph.
\newline This is another paragraph.

\section{Text formatting}
This is how to make text \textbf{bold}. This is how to
maek text \textit{italic} and this is how to make an
\underline{underline}.

\end{document}
